%% LyX 2.0.7dev created this file.  For more info, see http://www.lyx.org/.
%% Do not edit unless you really know what you are doing.
\documentclass[english]{article}
\usepackage[T1]{fontenc}
\usepackage[latin9]{inputenc}
\usepackage{amsmath}
\usepackage{setspace}
\onehalfspacing

\makeatletter
%%%%%%%%%%%%%%%%%%%%%%%%%%%%%% User specified LaTeX commands.
\usepackage{qtree}
\usepackage{tree-dvips}
\usepackage{ulem}
\usepackage{graphicx}
\newsavebox{\partbox}
\DeclareMathOperator*{\argmax}{arg\,max}

\makeatother

\usepackage{babel}
\begin{document}

\title{A Topical Collocation Model}


\author{Benjamin B�rschinger}


\date{09/12/2013}

\maketitle
These are some notes on a non-grammar version of Mark Johnson's Topic
Collocation Model. While equivalent to the original model, this re-formulation
allows for faster inference using a break-point sampler, rather than
having to perform actual parsing. In addition, the sampler we describe
might be applied to related uses of the Collocation-modeling idea,
e.g. in word-segmentation.


\section{The Topic Collocation Grammar}

Johnson (2010) introduced the Topic Collocation Grammar (TCG) that
is a non-parametric extension of LDA. Whereas the HDP (Teh et al.,
2006) uses non-parametric priors to allow for an unbounded number
of possible topics, the TCG uses non-parametric priors to allow for
an unbounded vocabulary, making it possible to learn semantically
coherent topical units such as ``electrical circuit board''.

It's original presentation makes use of Adaptor Grammars, an extension
of PCFGs which allows for a simple specification of certain models
by defining the generative process as that of generating a tree using
context-free rules.

\begin{align*}
\text{Base} & =\text{WordGen}(\Phi_{Base},p_{\#})\\
\theta_{i} & \sim\text{Dir}(\boldsymbol{\alpha}_{\theta})\\
\phi_{k} & \sim\text{DP}(\alpha_{\phi},\text{Base})\\
z_{i,j}\mid\boldsymbol{\theta} & \sim\text{Disc}(\theta_{i})\\
c_{i,j}\mid z_{i,j},\boldsymbol{\phi} & \sim\phi_{z_{i,j}}
\end{align*}


In words, we assume a base-distribution which can generate sequences
of words. We describe this distribution in more detail below although
we simply assume a Unigram process with fixed emission probabilities
$\Phi_{Base}$ and a fixed termination-probability $p_{\#}$. Fixing
these parameters simplifies implementation of the inference algorithm
but is not required. It is easy to put a Dirichlet prior on $\Phi_{Base}$
and a Beta on $p_{\#}$.

For each of the $N$ documents, we draw a document-specific mixture
over topics, one $\theta_{i}$ for each document $1\leq i\leq N$.
We also generate $K$ topics which are distributions over collocations,
that is, sequences of words. This is achieved by drawing each $\phi_{k}$
from a DP which takes as input the Base distribution.

Finally, for each of the ``units'' in each of the documents, we
generate a topic indicator from the document specific distribution
over topics and then the actual unit by using the corresponding topic
distribution over collocations.


\section{Inference}

Inference in this model is complicated by the fact that we do not
actually know the number of units in a given document as the actual
observations are not $c_{i,j}$s but the $w_{i,k}$s that make up
the collocations. This is illustrated by a toy example using 3 words
which, already, are compatible with 4 different hypotheses about the
extent of the $c_{i,j}$s, illustrated by bracketing.
\begin{itemize}
\item (electric circuit boards)$_{1}$
\item (electric)$_{1}$(circuit)$_{2}$(boards)$_{3}$
\item (electric circuit)$_{1}$(boards)$_{2}$
\item (electric)$_{1}$(circuit boards)$_{2}$
\end{itemize}
In addition to the actual grouping, the assignment to topics is also
latent, as in standard topic-models. Assuming a single topic, however,
we can immediately make the connection to the Goldwater Unigram model
of Word Segmentation model, equating words in the TCG case with characters
in the WS case. Goldwater introduced a conceptually simple and easy
to implement sampler for her Word Segmentation by realizing that each
possible hypothesis about the latent word-variables corresponds to
a binary vector of boundary indicators, and that it is easy to derive
a Gibbs-Sampler over this representation.

We will apply this idea directly to the TCG in the following paragraph.
We will focus on the case of a single document for simplicity.


\subsection{State-space}

The following latent-variables need to be performed inference for.
\begin{itemize}
\item the document specific distribution over topics $\theta$
\item the $K$ topic distributions $\phi_{k}$, all of which are draws from
(independent) DPs
\item the $N$ topic indicators $z_{i}$, where $N$ is the (latent) number
of collocations
\item the actual $N$ collocations $c_{i}$, where we constrain these such
that their linear concatenation gives rise to the actually observed
sequence of words $w_{1:N'}$ of $N'$ words
\end{itemize}
We integrate out $\theta$ and, crucially, all of the $K$ topic distributions,
giving rise to the well-known Posterior Predictive Distributions described
by the CRP. If the base-distribution for the DPs is fixed, we can
ignore the actual seating arrangements as we have a non-hierarchical
DP model in this case, simplifying book-keeping. It's straight-forward
to derive the added complications a non-fixed base-distribution raises
(see B�rschinger and Johnson (2011) for details).

The actual state-space of our sampler consists of $N'$ $K$-valued
variables that indicate the presence of a Collocation boundary, as
well as the Topic from which this collocation came. This is analogous
to the Goldwater (2007) sampler, except that rather than a binary
choice a $K$+1-ary choice needs to be made at every possible break-point.


\subsubsection{Example}

Assume $w_{1:4}=\text{red\,\ electric\,\ circuit\,\ boards}$, and
our current indicator vector is $b=\langle3,0,0,1\rangle$. This vector
encodes the following analysis

\[
(\text{red})_{3}(\text{electric circuit boards})_{1}
\]


That is, there are two collocations, one spanning the single word
``red'' and coming from Topic 3, and one spanning the words ``electric
circuit boards'' and coming from Topic 1. Let us now resample the
value for the first variable. Removing knowledge about it gives the
partial vector $b=\langle?,0,0,1\rangle$ which, just as in word segmentation,
requires us to be non-commital with all affected aspects of the previous
analysis. In fact, this affects the entire document as the next boundary
already is the end of the document. Consequently, we would need to
\begin{itemize}
\item remove one count for having used Topic 3
\item remove onecount for having used Topic 1
\item remove one count for having ``red'' in Topic 3
\item remove one count for having ``electric circuit boards'' in Topic
1
\item adjust the overall count of generated Topics / collocations (if there
were $n_{3}$ uses of Topic 3 before, now there are $n_{3}-1$; if
there were $n_{3,red}$ uses of ``red'' in Topic 3, now there are
$n_{3,red}-1$ such uses
\end{itemize}
Having made these adjustments, we can calculate the probability of
all possible ways of setting $?$. Setting it to $0$ woudl give rise
to the analysis

\[
(\text{red electric circuit boards})_{1}
\]


which differs from the original state in having this collocation.
Hence, we simply have to calculate

\[
P(\text{red electric circuit boards}|\text{Counts}_{1})P(\text{topic1}\mid\text{Counts})[P(\text{stop})]
\]


where the first factor corresponds to the probability of generating
the actual string from the Topic-1-CRP, given its current counts,
and the second factor is the probability of generating anything from
Topic1 in the current document, given the counts of Topic-uses. The
third factor is ``optional'' --- without it, we don't have a fully
generative model as we wouldn't be able determine the number of collocations.
It's impact is minor in word segmentation but make sense here, it
allows weak control over the number of collocations: setting termination
probability low encourages analyses with many short collocations,
setting it high encourages the use of few collocations.

Let's consider setting $?$ to any specific topic $x\in\{1..K\}$.
This gives rise to

\[
(\text{red})_{x}(\text{electric circuit boards})_{1}
\]


This time, the formula includes four more factors

\[
P(\text{red}\mid\text{Counts}_{x})P(x\mid\text{Counts})(1-P(\text{stop}))P(\text{electric...}\mid\text{Counts}_{1}+\{\text{red}\})P(1\mid\text{Counts }+\{x\})P(\text{stop})
\]


Note that we need to update the counts appropriately as the predictive
probabilities for the second collocation are affected by our decision
for the first collocation. Also note the additional (1-$P(\text{stop})$)
factor which adds additional cost for positing one more collocation.

That's it.
\end{document}
